\begin{cabstract}
随着社交媒体的日益普及,越来越多的人开始在网上实时地以各种方式表达自己的观点。社交媒体用户群体庞大,观点涉及话题广泛,使得网络成为能够挖掘出关于各种话题的大众观点的宝贵资源库。然而社交媒体中的观点常常是通过带有噪声的非结构化文本碎片中表达出的,并且这些碎片化文本分散在不同的来源(不同的用户),人工就某一话题去浏览所有的文本碎片并分析总结出相关的观点是非常困难的,需要以计算手段自动分析,整合并总结出所有文本中的观点信息。本文主要研究社交媒体用户观点的自动分析(包括观点挖掘和整合集成),主要目标是对用户在社交媒体上就所关注话题发表的大量观点更好地建模,并基于此模型进一步对用户的网络交互行为进行分析。

为了对问题进行系统地研究,本文确定了观点分析的三个主要步骤:情感知识的抽取,观点文本情感极性分类,用户观点的集成。这三个步骤组成了一个观点集成综合系统的三个关键组成部分,集成的用户观点信息促进了用户网络行为的分析研究。本文的主要贡献是对四个相互协同关联的观点分析与应用任务提出了新的通用的处理方法:

\begin{itemize}
\item \textbf{中文情感词典的抽取和构建}:目前表示情感知识的词典主要是在英文中构建的英文情感词典,这些词典在观点文本识别、极性分类等任务中起到了重要作用,是进行观点分析的基础。中文情感词典抽取和构建方法研究相对较少,还没有形成比较全面可靠的情感词典。靠人工编辑形成的情感词典费时费力,覆盖度偏低,因此本文根据不同语言间表达情感知识词汇间的对映性,借鉴已有的英文情感词典中的词语情感知识,使用HowNet语义知识库中词语的双语语义描述转化英文情感词典的情感知识,抽取中文情感词汇并计算情感极性值,形成了自动构建的中文情感词典SentiHowNet。为了提高词典的覆盖度以及领域情感知识的适应性,分析验证了基于语料资源中连词语言规则和上下文语境统计特征的情感词典扩展方法,并提出了混合两种方法的扩展方法对SentiHowNet在领域语料内进行了扩展。使用本文方法得到的中文情感词典可以自动构建无需人工标注,与其他几个词典相比覆盖度和领域适应性更好。
\item \textbf{基于特征空间划分的情感极性分类}:情感极性分类是按照文本中的特征共现规律将文本分类为特定的情感极性类别,可以看作是一种特殊的文本分类。情感极性分类最常用的词袋模型中,用以表达情感的词语特征常常起到不同的作用,有些词语具有通用情感表达作用,能在不同领域和语境中表示不变的情感极性,而有些词语只有在特定的领域和语境中才能表达特定的情感极性。因此本文提出了将特征空间划分为领域独立和领域依赖两部分的情感极性分类方法,该方法分别在两部分特征空间上训练分类器,然后将两个分类器组合在一个框架中形成一个更强的情感极性分类器,这种框架从现成的无需标注的资源开始,使用自举式的机器学习方法,可以在无需标注数据进行训练情况下达到有监督机器学习方法的性能。
\item \textbf{用户观点的集成建模}:社交媒体中用户产生的内容往往是短小而又分散的信息碎片,因此用户针对所关注话题的观点是碎片化在这些非结构化的短文本中。为了能够全面准确的了解用户的观点,本文提出了用户主观模型的概念,将用户产生内容中的所关注话题以及用户针对话题的观点组合在一起进行建模,模型中将观点按照话题的不同方面进行整合集成,并提出一种通用的可扩展观点表示方法,将同一话题的观点表示为在一个可扩展的情感值空间的分布,这种表示能够表达出用户更详细和多视角下的观点信息。
\item \textbf{用户交互行为分析}:作为用户主观模型的直接应用,本文对用户在社交媒体中信息传播行为的主观动机进行建模分析。针对Twitter中用户转发信息的三种常见情形,也就是用户对感兴趣和有吸引力的信息转发,用户基于社交需要对好友的信息转发以及用户对流行度高的信息转发,使用三个主观相似性计算方法进行度量。在转发行为的分析中,三种主观相似性度量与转发行为具有相关性,能够作为转发行为预测的有用特征,并能显著提高现有预测模型的性能。
%对用户的通过关注行为和其他用户链接以发现志同道合的好朋友这一主观动机,本文对有可能成为目标用户好友的用户群体进行主观相似性度量,并使用该度量指标进行分析,在实际Twitter数据上实验结果显示,相比于基于网络拓扑结构的一些指标,该指标更能准确解释用户以寻找好友为目的链接关注行为。
\end{itemize}

在对以上四个观点分析与应用任务的研究中,本文侧重于使用通用的鲁棒性好的无监督或弱监督方法,因此本文的方法适用于话题广泛的大量观点的自动分析,这也使我们的方法区别于针对特定领域精心进行特征设计并使用大量标注数据进行充分训练的有监督机器学习方法,因为这些方法转换到新领域就会变得性能下降,领域适应性差。我们尽可能使用现有的无需标注资源,比如一些现成的词典资源,可以为观点分析各种方法提供间接训练指导。基于这种思路我们的方法显示出良好的通用性并达到一定的评测性能,能够在多个研究领域(比如商业智能和社会学研究)得到应用。


\end{cabstract}
\ckeywords{社交媒体; 情感词典; 情感分类; 观点集成; 信息传播}

\begin{eabstract}

As Social Media becomes increasingly popular, more and more people express their opinions on the Web in various ways in real time. Such wide coverage of topics and abundance of users make the Web an extremely valuable source for mining people's opinions about all kinds of topics. However, since the opinions are usually expressed as unstructured noisy text fragments scattered in different sources(i.e., different users), it is difficult for the users to digest all opinions relevant to a specific topic within a large amount of text pieces, which needs the computational methods to automatically analyze, integrate and summarize the opinions articulated in all the text fragments. This thesis focuses on the problem of automatic opinion analysis including opinion mining, integration and summarization, whose goal is to better support modeling huge amounts of opinions for all topic of interests of social media users, and further to analyze their interaction behaviors based on these opinions. 

To systematically study this problem, we have identified three important steps of opinion analysis: extraction of sentiment knowledge, sentiment polarity classification of opinionate text, and opinion integration of users. These steps form three key components in an integrated opinion summarization system,the results of which are used to promote online behavior analysis of users. Accordingly, this thesis makes contributions in proposing novel and general computational techniques for four synergistic tasks: 

\begin{itemize}
\item \textbf{Extraction and construction of Chinese sentiment lexicon:} Current sentiment lexicons are built mainly for English sentiment knowledge, which are basis of opinion analysis and play important roles in tasks such as opinionative text identification and feature selection of sentiment classification, etc. There are relatively few studies on extraction and construction of Chinese sentiment lexicon, and there is no comprehensive and dependable Chinese sentiment lexicon available yet. The sentiment lexicon compiled by human is time-consuming and laborious, while has a low coverage. Therefore based on the sentiment knowledge mapping between words of different languages, and drawing from current English sentiment lexicons, we proposed a novel method to identify a number of Chinese sentiment words and calculate their sentiment polarity value using bi-linguistic semantic definition of HowNet knowledge resources, which formed a Chinese sentiment lexicon named SentiHownet. In order to improve coverage and domain adaptability of SentiHownet, we analyzed and verified language rules based extension method and corpus based statistical context features extension method with experiments, and proposed a hybrid method by combining two methods. The SentiHownet lexicon is constructed automatically without human annotation, which has wider coverage and better adaptability for domain opinion analysis than other Chinese sentiment lexicons.
\item \textbf{Sentiment polarity classification based on feature space division:} Sentiment classification classifies the text into predefined categories according to features co-occurence, and can be regarded as a kind of special text classification. The bag-of-words features of sentiment classification are often used with different functions: some features represent the same general sentiment polarity across different domains and context, while others represent specific sentiment polarity only in specific domain or context. Therefore, we proposed to divide the feature space of sentiment classification task into two separate parts, including domain-dependent part and domain-independent part. Two different classifiers are learned using two feature parts, and then combined together into a stronger sentiment polarity classifier in a bootstrapping framework.The framework started training on an off-the-shelf idiom resources without annotation in a bootstrapping way. The proposed method can achieve the performance of supervised methods without any annotation dataset.
\item \textbf{Integration of opinions of users:} User-generated content(UGC) of social media are often short and dispersed text fragments, so that the opinions of users about topic of interests are scattered in the unstructured fragmented short text. To be able to digest opinions of users comprehensively and accurately, we proposes the concept of subjectivity model by combining the topics and opinions together, in which the opinions are integrated according to the different aspects of the same topic articulated in the UGC. We also put forward a general representation of opinion, which defined opinion as sentiment distribution over a scalable sentiment value space, and provided a more detailed and informed multi-perspective view of the opinions.
\item \textbf{Interaction behaviors analysis of users:} As direct applications of subjectivity model, we analyzes the subjective motivation of the information dissemination behavior for the social media users. For three scenarios a Twitter user retweeted a message, that is, the user retweeted for he is interested and attracted by message content, the user retweeted a message of a close friend based on the social needs and the user retweeted for conformity needs because the message is popular, we proposed three subjectivity similarity measurements. For retweeting behavior analysis, the three subjectivity similarities are verified to be correlated to the retweeting behavior, and can serve as useful features for retweeting behavior prediction, which could significantly improve the performance of existing prediction models.
%As for following behavior analysis, the subjectivity similarities between every possible followee and the target user are calculated, the following behavior has been analyzed with the similarity measurement, and its effectiness in explaining the subjective motivation of the following behavior is veried on real Twitter data compared with measurements based on network toplogy.
\end{itemize}

We focus on general and robust methods which require minimal human supervision so as to make the automated methods applicable to a wide range of topics and scalable to large amounts of opinions. This focus differentiates this thesis from work that is fine-tuned or well-trained for particular domains but are not easily adaptable to new domains. Our main idea is to exploit many naturally available resources, such as off-the-shelf lexicon, which can serve as indirect signals and guidance for generating opinion analysis. Along this line, our proposed techniques have been shown to be effective and general enough to be applied for potentially many interesting applications in multiple domains, such as business intelligence and sociological Research.
\end{eabstract}
\ekeywords{Social Media; Sentiment lexicon; Centiment classification; Opinion integration; Information dissemination}

