\begin{cabstract}
随着社交媒体的日益普及,越来越多的人开始在网上实时地以各种方式表达自己的观点。这些观点覆盖各种话题,并且用户群体庞大,使得网络变成汇集了关于各种主题的大众意见的宝贵资源库。然而社交媒体中的观点通常是在充满噪声的非结构化的文本中表达的,人工就某一话题去阅读所有的和数据并提取总结出其中的观点是不可能的,需要以计算手段自动分析,整合集成,总结出文本中的观点信息。本文主要研究社交媒体用户观点的自动分析与整合集成,对用户在社交媒体上就所关注话题发表的大量观点进行综合建模,并对用户的网络交互行为进行分析。

为了对问题进行系统地研究,我们确定了观点分析的三个主要步骤:文本中情感知识的获取,文本情感倾向性分类,用户观点的集成建模。这三个步骤组成了一个观点集成汇总系统的三个关键组成部分,集成的用户观点信息促进了用户网络行为的分析研究。本文的主要贡献是对四个相互联系协同的观点分析与应用任务提出了新的通用计算方法:
\begin{itemize}
\item \textbf{中文情感词汇的抽取和情感词典的构建:}目前表示情感知识的词典主要是针对英文构建的情感词典,这些词典在观点信息识别、情感分类特征选择等任务中具有重要作用,是进行观点分析的基础。中文情感词典研究相对较少,还没有形成比较全面可用的情感词典,而靠人工编辑的情感词典费时费力,覆盖度偏低,因此本文首先根据不同语言间情感知识的对映性,借鉴已有的英文情感词典,使用HowNet语义知识库的语义关系计算确定了一些情感词汇并计算出情感极性值。为了提高词典的覆盖度以及领域情感知识的适应性,通过实验验证了基于语言规则和统计特征的基于语料库的情感词典的扩展方法,并提出了基于混合特征的扩展方法。
\item \textbf{基于特征空间划分的情感分类:}情感分类是按照文本中的特征共现规律将文本分类为特定的情感极性类别,是一种特殊的文本分类。用以表达情感的词语特征在情感分类任务中有不同的作用,有的词语就具有通用的情感表达能力,能在不同领域和语境中表示相同的情感极性,而有的词语只有在特定的领域和语境中才能表达特定的情感极性。因此本文提出了将情感分类的特征空间分为领域独立和领域依赖两部分,分别使用两部分特征训练分类器然后组合在一个框架中形成一个更强的情感分类器,这种框架从现成的无需标注的资源开始,使用自举式的机器学习方法,可以在无监督情况下达到有监督方法的性能。
\item \textbf{用户观点的整合集成:}社交媒体中用户产生的内容是短小而又分散的,因此用户针对某些话题的观点信息是碎片化在这些非结构化的短文本中。为了能够全面准确的了解用户的观点,本文提出了用户主观模型的概念,将用户产生内容中的话题信息以及用户针对话题的观点信息组合在一起,并将观点按照话题的不同方面进行整合集成,并提出一种通用的观点的表示方法,将同一话题的观点表示为在一个可扩展的情感空间的分布,这种表示能够提供用户更详细和多视角下的观点信息。
\item \textbf{用户转发行为分析:}作为用户主观模型的直接应用,本文对用户在在社交媒体中信息传播行为的主观动机进行建模分析。针对Twitter中用户转发信息的三种常见情形,也就是用户对感兴趣和有吸引力的信息转发,用户基于社交需要对好友的信息转发以及用户对流行度高的信息转发,使用三个主观相似性计算方法进行度量。在转发行为的分析中,三种主观相似性度量与转发行为具有相关性,能够作为转发行为预测的有用特征,并能显著提高现有预测模型的性能。
\end{itemize}

在对以上四个问题的研究中,我们侧重于使用通用的鲁棒性好的无监督或弱监督方法,因此我们的方法可以适用于话题广泛的大量观点的自动分析,这也是本文区别于一些针对特定领域精心进行特征设计并进行充分训练的其他方法,因为这些方法转换到新领域就会性能下降,领域适应性差。我们尽可能使用现有的无需标注资源,比如一些现成的词典资源,可以为观点分析各种方法提供间接训练指导。基于这种思路使得我们方法显示出良好的通用性和效能,能够在多个领域(比如商业智能和政治学)得到应用。


\end{cabstract}
\ckeywords{社交媒体; 情感词典; 情感分类; 观点集成; 信息传播}

\begin{eabstract}

As Social Media becomes increasingly popular, more and more people express their opinions on the Web in various ways in real time. Such wide coverage of topics and abundance of users make the Web an extremely valuable source for mining people's opinions about all kinds of topics. However, since the opinions are usually expressed as unstructured text scattered in different sources, it is difficult for the users to digest all opinions relevant to a specific topic within a large amount of text pieces, which needs the computational methods to automatically analyze, integrate and summarize the opinions articulated in the text. This thesis focuses on the problem of opinion integration and summarization whose goal is to better support digestion of huge amounts of opinions for an arbitrary topic and model the interaction behavior of users. To systematically study this problem, we have identified three important steps of opinion analysis: extraction of sentiment knowledge, sentiment classification of text, and integration of opinions. These steps form three key components in an integrated opinion summarization system. Accordingly, this thesis makes contributions in proposing novel and general computational techniques for four synergistic tasks: 
\begin{itemize}
\item \textbf{Extraction and construction of Chinese sentiment lexicon:}Current sentiment lexicons are built mainly for English sentiment knowledge, which are basis of opinion analusis and play important roles in such tasks as subjectivity analysis, feature selection of sentiment classification, etc. There are relatively few studies on construction of Chinese sentiment lexicon, and there is no comprehensive lexicon available yet. However the sentiment lexicon compiled by human is time-consuming and laborious, while has a low coverage. Therefore based on the sentiment knowledge mapping between different languages and current English lexicons, we proposed a novel method to identify a number of positive and negative words and calculate their sentiment strength value using semantic relationships of HowNet semantic knowledge dictionary. In order to improve coverage and domain adaptability of sentiment lexicon, we verified language rules based and corpus based statistical extension methods with experiments, and proposed a hybrid features method.
\item \textbf{Sentiment classification based feature space division:}Sentiment classification classifies the text into predefined categories according to feature co-occurence, and can be regarded as special text classification. The features of sentiment classification are used in different position: some features represent the same general sentiment  polarity across different domains and context, while others represent specific sentiment polarity only in specific domain and context. Therefore, we proposed to divide the feature space of sentiment classification into separate parts, which are domain-dependent part and domain-independent part. Two different classifiers are learned using two feature parts, and then combined together to form a stronger sentiment classifier in a bootstrapping machine learning frame, which started training on an off-the-shelf resources without annotation in an unsupervised bootstrapping way. The bootstrapping method can achieve the performance of supervised methods in an unsupervised situations.
\item \textbf{Integration of opinions of users}User-generated content of social media is short and dispersed, so that the opinions of users about certain topic are scattered in the unstructured fragmented short text. To be able to understand comprehensively and accurately the users' opinions, we proposes a subjectivity model to combine the topics and the opinions integrated according to the different aspects of the same topic articulated in UGC. We also put forward a general representation of opinion, which defined opinion as sentiment distribution over a scalable sentiment space, and provided a more detailed and informed multiperspective view of the opinions.
\item \textbf{Retweeting analysis of users:}As a direct application of subjectivity model, we analyzes the subjective motivation of the information dissemination behaviors for the social media users. For three scenarios a Twitter user retweeted a message, that is, the user retweeted for he is interested and attracted by message content, the user retweeted a message of a close friend based on the social needs and the user retweeted because the message is popular, we proposed three subjectivity similarity measurements. For retweeting behavior analysis, the three subjectivity similarities correlated to the behavior, and could serve as a useful features for retweeting behavior prediction, which could significantly improve the performance of existing prediction models.
\end{itemize}

We focus on general and robust methods which require minimal human supervision so as to make the automated methods applicable to a wide range of topics and scalable to large amounts of opinions. This focus differentiates this thesis from work that is fine-tuned or well-trained for particular domains but are not easily adaptable to new domains. Our main idea is to exploit many naturally available resources, such as off-the-shelf lexicon, which can serve as indirect signals and guidance for generating opinion summaries. Along this line, our proposed techniques have been shown to be effective and general enough to be applied for potentially many interesting applications in multiple domains, such as business intelligence and political science.
\end{eabstract}
\ekeywords{Social Media; Sentiment lexicon; Centiment classification; Opinion integration; Information dissemination}

