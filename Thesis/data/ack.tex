
%%% Local Variables:
%%% mode: latex
%%% TeX-master: "../main"
%%% End:

\begin{ack}
  时间应该“浪费”在美好的事物上,我很庆幸自己有机会从事这些课题的研究。
  
本课题承蒙国家自然科学基金-面上项目“融合网络特征的文本观点挖掘”(项目号:61170156)和国家自然科学基金-青年科学基金项目“结合社会网络的网络信息传播分析研究”( 项目号:61202337)的资助,另外感谢国家留学基金委对本人在英国爱丁堡大学留学访问的经费资助。经济基础决定上层建筑,没有钱一切白搭。

感谢我的导师王挺教授对本人的精心指导和悉心培养,您严谨治学的态度让我受益终生。感谢在我英国爱丁堡大学留学期间指导我的Miles Osbonre博士,您对科研的敏锐洞察力以及极强的逻辑思维能力让我明白科研可以轻松地“玩”。

感谢国防科技大学自然语言处理组的张晓艳、刘伍颖、唐晋韬、魏登萍、周云、李岩、麻大顺、谢松县、刘培磊、岳大鹏、刘海池、汝承森、张文文、姜仁会、胡长龙、李欣奕,和你们一同探索自然语言处理的未知领域让人回味;感谢英国爱丁堡大学信息学院348办公室的Saša Petrovic、Desmond Elliott、Diego Frassinelli、Eva Hasler、Michael Auli和Luke Shrimpton,和你们仔细讨论我的课题细节以及英文的论述让我十分收益。

感谢戴波、雷鸣、林正帅、毛先领、张湘莉兰、任洪广、王鹤、吴诚堃标注Twitter观点检索相关数据,感谢王铮标注Twitter传播观点检索相关数据。没有你们的无偿帮助,我相信我的课题研究不会如此顺利。感谢Victor Lavrenko博士和Micha Elsner博士给予我Twitter观点检索课题宝贵的意见,和你们讨论是我的荣幸。

感谢我从硕士到博士的室友邹丹、何明、陆化彪,一起研究课题慢慢“变老”的过程值得怀念;感谢我在爱丁堡期间最好的朋友王鹤和黄轩,人生得不多的知己足以;感谢在英国一同留学的杨俊刚、雷鸣、Chee-Ming Ting、陆亮、王铮、刘哲、马瑞、冯翌尧、卢恒、林正帅、曾旋、杨国利、何鑫、胡尽力、贺建森、杨丽莎、魏杰、罗家希,谢谢你们让我拥有那些留学的“回忆”,谢谢你们陪我那时的“孤独”。

感谢我的父母、奶奶、岳父、岳母对我生活上无微不至的照顾;最后感谢我的爱人柳意,谢谢你的支持,爱你!
\end{ack}
