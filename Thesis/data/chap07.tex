\chapter{总结与展望}
\label{con}
社交媒体已经逐步发展完善,随着用户使用社交媒体的普及,带有用户观点信息的文本数据正爆炸式增长,本文主要围绕社交媒体中观点信息的挖掘、分析以及在行为分析中的应用展开研究。通过情感词典资源的建设、情感极性分类以及利用情感分析结果对社交媒体用户的主观性建模和应用等任务,我们充分利用了社交媒体作为媒体所产生的文本特点和社交媒体用户之间的社交功能来帮助解决这些问题。

对社交媒体中文本数据中的观点信息分析研究能够从社交媒体海量数据中发现有借鉴意义的信息,无论对于其他研究还是商业应用都有价值。为了能够得到观点信息需要从文本中抽取分离出能够识别用户看法、态度、立场以及情感的表达方式,我们特别针对社交媒体的文本进行了情感知识辞典的构建和对社交媒体非规范化文本的情感分类研究,因此可以从数据中挖掘分析观点信息。

在得到文本中的观点信息后,可以帮助我们来认识作为社交媒体使用主体的用户,对用户在社交媒体上表达的观点进行集成分析,对用户的主观性进行建模。得到的主观模型对于我们理解用户的在线行为分析提供帮助。

\section{工作总结}

本文的主要工作可以从以下五个方面来总结:

首先,针对现有中文情感词典相对较少并且缺乏标准性问题,我们提出了借鉴现有的丰富的英文情感词典资源进行跨语言的情感知识转化研究。为了更准确的反映词语的情感情感极性,我们结合中文语义知识库HowNet,将知识库中的语义关系融合进词语的情感值计算过程中,利用HowNet的义原与词语的中英文对应关系见英文词典SentiWordnet的情感知识转化为中文词语的情感知识,形成中文情感词典SentiHowNet。

第二,仅仅依靠从词典资源标注或转化的情感知识是不够的,社交媒体语言的动态性决定需要一种能够及时从社交媒体语料数据中发现新的情感知识词典的方法,我们通过研究中文的语言规则和统计特征,实验验证从语料中抽取词语并计算情感值形成新的情感词典方法,使得情感词典可以适应社交媒体语言不断增长与变化的特点得以及时补充情感知识。

第三,从社交媒体文本尤其是一些不规范的短文本中发现观点信息需要对文本的情感倾向性进行分类,通过将情感分类问题形式化为特殊的文本分类问题,我们提出了特征空间划分的假设,并使用现成的无须标注成语资源在不同的特征空间训练通用的分类器,并使用自举式机器学习方法解决了社交媒体的文本缺少大规模的标注文本而无法训练分类器的问题。

第四,用户在使用社交媒体时发表的信息是短小的、碎片化的,因此用户的观点信息散布在这些碎片化的文本中。以往的情感分析只是针对文本片段分析得出文本片段的观点信息,无法完整呈现出一个用户整体的观点,我们首先提出用户观点集成问题,并就这一问题提出了主观模型框架。在主观模型中,我们将用户在社交媒体中感兴趣的话题以及针对这些话题的观点进行了集成,并提出了观点的一种通用的表示形式,将观点表示为在情感空间的一种分布。主观模型可以对用户在社交媒体中的观点信息集成表示,解决了用户信息的碎片化而造成的表示不全面准确问题。

最后,针对信息传播研究中忽略的人的传播主观动机问题,结合主观模型对用户的主观性的建模分析,我们研究了Twitter中用户的转发行为,以此帮助理解Twitter中信息是如何传播的。我们将用户转发行为的主观动机量化为主观相似性,通过分析影响用户转发行为的三个因素,也就是内容上的吸引力,朋友间的社交需要以及微博信息的流行性,转化为三个主观相似性度量并分析研究了它们与转发行为之间的关系,实验证明了主观相似性度量与转发行为的相关性,以及在预测转发行为的有效性。

\section{工作展望}
展望未来,社交媒体中的观点信息的分析研究及其相关方向还有很多工作需要完成。这里总结以下亟待探索的研究方向和路线:
  \begin{enumerate}
  \item 以Twitter为代表的社交媒体一个重要特点就是消息的实时性,许多研究工作都围绕在Twitter中发现实时信息展开,包括新事件发现\upcite{petrovic2010streaming,becker2011beyond,weng2011event,naaman2011hip,benson2011event,petrovic2012using,kanhabua2013understanding}、实时灾害报道(如地震、疾病、火灾等)\upcite{sakaki2010earthquake,paul2011you,aramaki2011twitter,abel2012twitcident,yin2012esa},另外,TREC的Twitter检索\upcite{efron2011estimation,metzler2012structured,zhang2012query,soboroff2012evaluating,choi2012temporal,amati2012survival,miyanishi2013combining}也将实时性作为一个重要指标。本文的研究中,我们并未对观点挖掘与分析受到实时性的影响进行讨论研究\upcite{Das2014,Guerra2014}。
  
 \item 本文的社交媒体中观点信息的研究还是对比较常用的几个类型(比如评论和微博)进行的研究,实际上社交媒体还有很多类型,如Facebook\footnote{\url{https://www.facebook.com/}}、YouTube\footnote{\url{https://www.youtube.com/}}、 Flickr\footnote{\url{https://www.flickr.com/}}等等。这些社交媒体肯定有自己独特的特点,在其数据上进行观点的挖掘与分析需要研究其独特的情感表达方式;另外,多种社交媒体综合和跨媒体的信息交互与传播也会对观点信息的分析提出新课题,这就需要研究者在充分理解各种社交媒体的特点和人们对各种社交媒体不同使用习惯上,提出方法解决问题。
 
 \item 目前将观点分析结合进其他应用和任务研究是一个新的研究方向,主要是基于目前越来越多的应用和任务需要以社交媒体用户的观点作为有用的特征来使用,比如在股市指数的预测中人的情感指标会影响者投资意愿,未来我们见结合更多的和实际任务或应用有针对性的进行观点信息的分析。 
  \end{enumerate}
  
  总之,针对社交媒体中的观点信息的研究还有许多问题等待着去解决,我们将继续深入研究相关问题。

